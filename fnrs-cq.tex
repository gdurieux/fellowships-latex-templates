\documentclass[12pt,a4paper]{article}
\usepackage[T1]{fontenc} 
\usepackage[utf8]{inputenc}
\usepackage[UKenglish]{babel}
\usepackage{lmodern, amsmath, xcolor, rotating, multirow, adjustbox}
\usepackage[numbers,elide,sort&compress]{natbib}
\usepackage[bottom=1.65cm, top=1.5cm, left=1.5cm, right=1.5cm]{geometry}
\usepackage[colorlinks=true, citecolor=blue, urlcolor=blue, linkcolor=blue, breaklinks=true,pdfpagelabels=false ]{hyperref}
\usepackage[normalem]{ulem}

\renewcommand{\familydefault}{\sfdefault}
\usepackage{sfmath, sansmathaccent}


% tikz
\usepackage{tikz}


\makeatletter

\renewcommand{\thesection}{\arabic{section}.\hspace*{-.4cm}}
\renewcommand{\thesubsection}{\arabic{section}.\arabic{subsection}\hspace*{-.2cm}}
\setlength{\parindent}{0pt}
\setlength{\parskip}{7pt plus 5pt minus 3pt}
%\setlength{\headsep}{0mm}
\setlength{\footskip}{1cm}
\setlength{\headheight}{4.5mm}

\renewcommand{\section}{\@startsection{section}{1}{\z@}{10pt}{1pt}{\Large\bfseries}}
\renewcommand{\subsection}{\@startsection{subsection}{2}{\z@}{7pt}{1pt}{\large\bfseries}}
\renewcommand{\@oddhead}{%
\small\hfill%
	\begin{tabular}{@{}r@{}}
	\\[-2.5mm]
	F.R.S.-FNRS GRANTS AND FELLOWSHIPS CALL 20xx\\[-.75mm]
	`Research Associate' (CQ - Chercheur qualifié)
	\end{tabular}%
}
\newcommand{\instructions}[1]{{\small\it\fontsize{11.5pt}{.20\baselineskip}\selectfont\parbox{\textwidth}{#1}}\nopagebreak\par}

\newenvironment{myitemize}{\list{--}{\topsep 0pt \parsep 0pt \itemsep 0pt\itemindent 0\labelwidth \leftmargin 1\labelwidth}}{\endlist}
\newenvironment{mybigitemize}{\par\list{}{\topsep 0pt\itemsep 0pt\parsep 0pt\leftmargin 0pt \itemindent 0\labelwidth}}{\endlist}

\newcommand{\rien}[1]{}


\makeatother
\begin{document}
\large
\fontsize{13.3pt}{.77\baselineskip}\selectfont
%------------------
{\centering\begin{small}
\fontsize{10.7pt}{.85\baselineskip}\selectfont%
{\renewcommand{\arraystretch}{1.2}
\vspace*{-3.5mm}
\begin{tabular}{@{}|p{.4775\textwidth}|p{.4775\textwidth}|@{}}
\hline Full name of the applicant &
	%
	\\
\hline Reference &
	%
	\\
\hline
\end{tabular}
}\vspace{.5cm}

\MakeUppercase{\bfseries\Large Scientific section of the proposal}
\par\MakeUppercase{\small Main language chosen = English}
\vspace{4mm}

\hfil\begin{tabular}{|p{10.75cm}|}
\hline
\vspace{1sp}
This part includes the following elements:
\begin{list}{\arabic{enumi}.}{
	\usecounter{enumi}
	\itemsep 0pt \parsep 0pt \parskip 0pt
	\settowidth{\labelwidth}{10.}
%	\setlength{\itemindent}{-\labelsep}
%	\addtolength{\itemindent}{-\labelwidth}
	\setlength{\leftmargin}{1.3cm}
%	\setlength{\itemindent}{\labelsep}
%	\addtolength{\itemindent}{\labelwidth}
}
\item Report on past research
\item Description of the research project
\item Comments on changes made in the research project in case of resubmission (optional)
\item Potential interdisciplinary approach of the research project (optional)
\item Description of the work environment
\item Degree of independence
\item Publications of the applicant related to the proposal
\item Five most significant publications of the applicant
\item Comments on mobility (optional)
\item Additional comments (optional)
\end{list}

\vspace{2mm}
\textbf{\color[RGB]{31,73,125}The applicant must fill in the sections below and convert the file into an unprotected PDF before appending it to the online application form.}

\vspace{4mm}
The F.R.S.-FNRS insists on \textbf{strict compliance with the instructions given for each part of the proposal} (scientific section relevant to the instrument selected, number of pages allowed for documents to be enclosed with the application form...) and stresses again the sovereign consideration of the Scientific Commissions assessing the application file.
\\[1.5mm]\hline
\end{tabular}
\end{small}
}


\section{Report on past research}
\label{sec:past-work}
\instructions{Please provide a report underlining your past research
activities and possible achievements (max.\ 2~pages).}


\section{Description of the research project}
\label{sec:proj}
\instructions{The written project (max.\ 4 pages) according to the structure below, accompanied by a reference bibliography (max.\ 1 page besides the 4 pages dedicated to the project) listed by order of appearance within the text.
Graphs and tables may be added (max.\ 2 pages).}

\subsection{Goals of the research}


\subsection{State of the art} 


\subsection{Research project}


\subsection{Work plan}


\section{Comments on changes made in the research project in case of resubmission (optional)}
\label{sec:resub}
\instructions{In case of former application submitted to the F.R.S.-FNRS via the
same funding instrument, please specify the main changes made in your funding
application following previous submission, identifying comments from experts
that you may have taken into account (max.\,1 page).}


\section{Potential interdisciplinary approach (optional)}
\label{sec:interdis}
\instructions{If applicable, please identify the interdisciplinary approach of
your research project (max.\ 1 page).}


\section{Description of the work environment}
\label{sec:work-env}
\instructions{Please provide the information accounting for the adequacy of the environment (available intellectual and/or material means) to carry out the research as detailed in the submitted project.
Please specify the assets of the research environment related to the project and the main publications of the laboratory/promoter (max.\ 1 page).}


\section{Level of independence}
\label{sec:indep}
\instructions{Please provide the information you deem relevant to your level of independence and / or the way you plan to gain independence in the coming years (max.\ 1 page).}



\section{Publications of the applicant related to the proposal}
\label{sec:proposal-pubs}
\instructions{Please provide a selection of maximum 3 of your most relevant
publications related to the subject of your research project, including an
abstract or a summary.}



\section{Five most significant publications of the applicant}
\label{sec:top-pubs}
\instructions{Please provide a selection of 5 of the most significant
publications in your career, including an abstract or a summary.\phantom{tf}}



\section{Comments on mobility (optional)}
\label{sec:mobility}
\instructions{Under this section you may indicate any information relevant to your past, current or expected mobility. In this case, please underline the importance of this mobility for conducting the project presented. On the other hand, you may comment on the absence of mobility (max.\ 1 page).}


\section{Additional comments (optional)}
\label{sec:comments}
\instructions{If you want to communicate elements that have not been mentioned
elsewhere in the file, please provide this information below in max.\ 2 pages.

Please note that in case the presented project provides for the involvement of patients and/or human or animal subjects, it is important that the project includes justifications on the planned sample size (number of subjects included in the study/studies) and how the size is relevant (based on statistical power calculations, for instance). It is also important to explain how the number of patients/subjects expected can be reached. In case the project provides for the involvement of patients and/or subjects, please provide those pieces of information under this section (if not already mentioned elsewhere in the project). Ultimately, this information (or the lack of information) may be taken into account by experts in the frame of the evaluation of your funding application.
}


\bibliographystyle{apsrev4-1_title}
%\bibliographystyle{apsrev4-1}
{\footnotesize\raggedright\setlength{\bibsep}{.5\parskip}\bibliography{references}}
\end{document}
